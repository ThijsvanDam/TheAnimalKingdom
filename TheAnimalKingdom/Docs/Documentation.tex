% Preamble
\documentclass[11pt]{article}

% Packages
\usepackage{a4wide}
\usepackage{graphicx}
\usepackage{url}
\graphicspath{{images/}}
\usepackage[nottoc]{tocbibind}
\usepackage{listings}

% Author
\author{Van Dam, Thijs\\
\and
Bos, Sander\\
}
\title{\huge The Animal Kingdom}
\date{April 2018}

\setlength{\parindent}{0em}
\setlength{\parskip}{1em}

% Document

\begin{document}
    \maketitle
    \thispagestyle{empty}
    \newpage
    \newpage
    \setcounter{page}{1}
    \section{Introduction}\label{sec:introduction}
    \begin{itemize}
        \item Steering
        \item Path planning
        \item Behaviour
        \item Fuzzy logic
    \end{itemize}

    \newpage
    \tableofcontents
    \newpage
    %-----------------------------------------------------------------------------------------
    \section{Steering}\label{sec:steering}

    \newpage
    %-----------------------------------------------------------------------------------------
    \section{Path planning}\label{sec:pathPlanning}
    This section contains all information about the path planning algorithms and structure used in 'The Animal Kingdom'.
    \subsection{Structure}\label{subsec:pathstructure}
    Buckland\cite{pgaie} describes a well-structured approach to splitting logic for the pathfinding.
    This same structure is what we used in our application, although we have made some changes to fit our own needs.
    The application contains one instance of a \textbf{PathManager} and each (moving) entity contains a \textbf{PathPlanner}.
    The \textbf{PathPlanner} is what is used to request a new search with the specified algorithm (A* or Dijkstra).
    When a request is done, the \textbf{PathPlanner} registers itself in the \textbf{PathManager}.
    Using a \textbf{PathManager} makes it possible to manage and control all search requests from one place.
    This is especially useful when you want to implement time-slicing or other logic that restricts the amount of search cycles per update.
    The latter case is what we use it for.
    \textbf{PathManager} contains a function \textit{UpdateSearches} which lets each \textbf{PathPlanner} do one or more cycle depending on the maximum cycles set while instantiating the \textbf{PathManager}.
    What a cycle does, will be discussed in section \ref{subsec:pathalgorithms}.\par
    The class diagram below shows the structure of the pathfinding logic including the \textbf{PathManager} and \textbf{PathPlanner}.
    
    \subsection{Creating a graph}\label{subsec:pathgraphcreation}
    
    \subsection{Algorithms}\label{subsec:pathalgorithms}
    
    \subsubsection{Dijkstra}\label{sec:pathDijkstra}

    
    \subsubsection{A*}\label{sec:pathAstar}
    
    \subsection{Performance}\label{subsec:pathperformance}
    
    \newpage
    %-----------------------------------------------------------------------------------------
    \section{Behaviour}\label{sec:behaviour}
    
    \newpage
    %-----------------------------------------------------------------------------------------
    \section{Fuzzy logic}\label{sec:fuzzyLogic}

    \newpage
    %-----------------------------------------------------------------------------------------
    \section{Conclusion}\label{sec:conclusion}

    \newpage
    %-----------------------------------------------------------------------------------------

    \bibliographystyle{unsrt}
    \bibliography{sources}
\end{document}