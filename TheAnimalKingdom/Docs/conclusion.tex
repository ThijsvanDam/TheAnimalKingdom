\subsection[Reflection on the end result]{Reflecion}\label{subsec:reflection}
Looking back at this project I think we did a pretty good job.
Since it was a school exercise we wanted to get it done really fast, and sometimes the development lacked of some testing.
Yet, we tested it good enough to pass working parts of the program to eachother.
So, actually, the not testing was not really in our way!
It mainly left some insecurities and we had to hope for the code to work.

Then we wanted to implement everything on the best, neatest possible way and I think that was in our way sometimes.
This way we lost a lot of time thinking things through that actually were finished already.

In the beginning we did a lot of paired programming and I think this cost a lot of time as well.
Mostly since you simply do not have the producing code of two programmers, but of only one.

Looking at these point, you may notice that there was a severe lack of time.
We actually spent a lot of hours on this project, but it was really hard work to get it done in time.

\subsection[Improvements of our project]{Improvements}\label{subsec:improvements}
There are some big improvements that can be done to our game.
This is mainly aimed at the looks of the game, the use of sprites and the behaviours of the animals.

Currently we only used images and not really sprites to represent our entities, this could be a lot better if we just used sprites!
Once again, this was mainly because the lack of time we had.

Then the behaviour of the animals.
With this I mean that we had a limit to how many fuzzy modules / sets we could implement.
The currently implemented fuzzy logic only handles if a gazelle wants to eat or to run from a lion.
The lion cannot even decide if it goes after a gazelle yet!
That is a big bummer, because it would be a lot more interactive if this did work.

We also did not take into account that entities wont flee from another entity if there is an obstacle inbetween.
Right now, we just run from the other entity no matter what.
This would be a pretty easy, but quite essential modification.
Mostly since the gazelle would actually hide behind something and this would look pretty realistic.

The last big improvement I can think of, is some better implementation of the FuzzyLogic into the GoalManager.
Right now we just created a static FuzzyManager to get all configurations for the FuzzyLogic.
We could have implemented this inside the GoalManager much more beautiful (for \textbf{instance} (hehe..) a non-static class).
